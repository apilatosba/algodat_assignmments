\documentclass{article}
\usepackage{amsmath, amsthm, amssymb}
\usepackage{tikz}
\usepackage{array}
\usepackage{mathtools}
\usepackage{enumitem}
\usepackage{algorithm}
\usepackage{algpseudocode}

\begin{document}

\section*{\huge Homework Sheet 8}
\begin{flushright}
  \begin{tabular}{@{} l l l @{}}
    \textbf{Author} & \textbf{Matriculation Number} & \textbf{Tutor} \\
    Abdullah Oğuz Topçuoğlu         & 7063561 & Maryna Dernovaia \\
    Ahmed Waleed Ahmed Badawy Shora & 7069708 & Jan-Hendrik Gindorf \\
    Yousef Mostafa Farouk Farag     & 7073030 & Thorben Johr \\
  \end{tabular}
\end{flushright}

% --------------------------------------------------------------------
% ----------------------------- EXERCISE 1 ---------------------------
% --------------------------------------------------------------------

\section*{Exercise 1 (Coin Flips)}

\begin{enumerate}[label=\textbf{\alph*.}]

% (a)
\item \textbf{}

This is deterministic. She always stops after exactly $n$ flips.
\[
\mathbb{E}[T] = n.
\]

% (b)
\item \textbf{}

Let $T$ be the time of the first head. This is a geometric random variable with success probability $p=\tfrac12$:
\[
\mathbb{E}[T] = \frac{1}{p} = 2.
\]

% (c)
\item \textbf{}

This is the negative-binomial stopping time for $n$ successes with success probability $p=\tfrac12$:
\[
\mathbb{E}[T] = \frac{n}{p} = 2n.
\]

% (d)
\item \textbf{}

Let $E_0,E_1,E_2$ denote the expected remaining flips given that the current run of consecutive tails has length $0,1,2$.  
We stop when we reach a run of $3$ tails, so $E_3=0$.

We set up recurrences:
\[
\begin{aligned}
E_0 &= 1 + \tfrac12 E_1 + \tfrac12 E_0,\\[4pt]
E_1 &= 1 + \tfrac12 E_2 + \tfrac12 E_0,\\[4pt]
E_2 &= 1 + \tfrac12 \cdot 0 + \tfrac12 E_0.
\end{aligned}
\]

Solving these:
\[
E_2 = 1 + \tfrac12 E_0, \qquad
E_0 = 2 + E_1.
\]
Substitute into the equation for $E_1$:
\[
E_1 = 1 + \tfrac12(1 + \tfrac12 E_0) + \tfrac12 E_0
     = \tfrac32 + \tfrac34 E_0.
\]
Thus,
\[
E_0 = 2 + \tfrac32 + \tfrac34 E_0
\quad\Rightarrow\quad
E_0 - \tfrac34 E_0 = \tfrac74
\quad\Rightarrow\quad
E_0 = 14.
\]

Hence the expected number of flips until three consecutive tails appear is
\[
\mathbb{E}[T] = 14.
\]

\end{enumerate}

% --------------------------------------------------------------------
% ----------------------------- EXERCISE 3 ---------------------------
% --------------------------------------------------------------------

\section*{Exercise 3}

\begin{verbatim}
void Permute(A[1..n])
   if (n == 1) return
   int randomIndex = rand(n) // rand() function from the lecture
   swap(A[1], A[randomIndex])
   Permute(A[2..n])
\end{verbatim}

\textbf{Correctness Proof:} \\
We prove by induction that the algorithm produces a uniform random permutation of the array $A[1..n]$.

\textbf{Base Case:} For $n = 1$, there is only one permutation. The algorithm returns the array unchanged.

\textbf{Inductive Step:} Assume the algorithm produces a uniformly random permutation for arrays of size $k-1$.  
For size $k$, the algorithm selects a random index from $1$ to $k$, placing each element in the first position with probability $1/k$.  
Then it recursively permutes the remaining $k-1$ elements uniformly by the induction hypothesis.  
Thus the resulting permutation is uniform.

\textbf{Running Time:}
\[
T(1)=1,\qquad T(n)=T(n-1)+1,
\]
so the running time is
\[
T(n) = O(n).
\]

\end{document}
