\documentclass{article}
\usepackage{amsmath, amsthm, amssymb}
\usepackage{tikz}
\usepackage{array}
\usepackage{mathtools}
\usepackage{enumitem}
\usepackage{algorithm}
\usepackage{algpseudocode}

\begin{document}
\section*{\huge Homework Sheet 7}
\begin{flushright}
  \begin{tabular}{@{} l l l @{}}
    \textbf{Author} & \textbf{Matriculation Number} & \textbf{Tutor} \\
    Abdullah Oğuz Topçuoğlu         & 7063561 & Maryna Dernovaia \\
    Ahmed Waleed Ahmed Badawy Shora & 7069708 & Jan-Hendrik Gindorf \\
    Yousef Mostafa Farouk Farag     & 7073030 & Thorben Johr \\
  \end{tabular}
\end{flushright}

% Please note: An exercise that asks you to design or implement an algorithm is to be answered
% by (1) your algorithm in pseudocode, (2) a correctness proof, and (3) a running time analysis.
% Exercise 1 (Recurrences) 2+2+3+3+3 points
% For each of the following functions defined by a recurrence inequality, give a closed (asymptotic) form
% by applying the Master theorem. In each case, specify the parameters a, b, d (and s), and state which
% of the three cases of the Master Theorem presented in the lecture applies.
% a. f(n) ≤ 5 · f(n/2) + n
% 2
% ,
% b. g(n) ≤ 9 · g(n/3) + n
% 2
% ,
% c. h(n) ≤ 2 · h(n/3) + n log2 n.
% d. k(n) ≤ 21 · k(n/9) + n
% 1.5√
% log n.
% e. ℓ(n) ≤ 3 · ℓ(n/9) + n
% 0.1
% log n,
% Exercise 2 (Mysterious Divide-and-Conquer) 4+3 points
% Consider the following divide-and-conquer algorithm Mystery, which takes as input an integer
% array A of length n. The procedure Concat concatenates two given arrays in linear time.
% 1: procedure Mystery(A[1 .. n]):
% 2: if n ≤ 1 then return true
% 3: if n = 2 then
% 4: if A[1] ≤ A[2] then return true else return false
% 5: Split A into three subarrays A1, A2, A3 of lengths ⌊
% n
% 3
% ⌋, ⌊
% n
% 3
% ⌋, n − 2⌊
% n
% 3
% ⌋
% 6: B1 := Concat(A1, A2)
% 7: B2 := Concat(A1, A3)
% 8: B3 := Concat(A2, A3)
% 9: return Mystery(B1) ∧ Mystery(B2) ∧ Mystery(B3)
% a. Find out what Mystery(A) computes and prove your claim.
% b. Use the Master theorem to analyze the running time of the algorithm.
% 1
% Exercise 3 (Local Minimum) 10 points
% You are given an integer array A[1..n] with A[1] ≥ A[2], A[n − 1] ≤ A[n], and n ≥ 3. Your task is to
% find a (non-strict) local minimum of A: compute an index i ∈ {2, . . . , n − 1} such that A[i − 1] ≥ A[i]
% and A[i] ≤ A[i + 1]. Design an algorithm to solve this task in time O(log n).
% Hint: Use Divide and Conquer. Split A in the middle.
% Exercise 4 (Multiplication) 10 points
% You are given an integer array A[1..n], where each number fits into one memory cell. Design an
% algorithm that computes the product of all numbers in A in time O(n
% log2
% (3)) ⊂ O(n
% 1.585). For this
% task, you may not assume that all intermediate values fit into one memory cell.
% Hint: The product of i numbers, each fitting into one memory cell, can be stored in i memory cells.

\section*{Exercise 1}
Remember the master theorem
\begin{align*}
   \text{If } \quad &T(n) \leq a \cdot T\left(\frac{n}{b} + r\right) + c \cdot n^d \log^sn \\
   \text{ then } \quad &T(n) = \begin{cases}
      O(n^d \log^s n) & \text{if } a < b^d \\
      O(n^d \log^{s+1} n) & \text{if } a = b^d \\
      O(n^{\log_b a}) & \text{if } a > b^d \\
   \end{cases}
\end{align*}

\subsection*{(a)}
We are given the recurrence
\begin{align*}
   f(n) &\leq 5 \cdot f\left(\frac{n}{2}\right) + n^2
\end{align*}

Values of paramters are
\begin{align*}
   a &= 5 \\
   b &= 2 \\
   d &= 2 \\
   s &= 0
\end{align*}

Since \( a > b^d \) that is \( 5 > 2^2 \) the third case applies.
\begin{align*}
   f(n) &\in O\left(n^{\log_2 5}\right)
\end{align*}

\subsection*{(b)}
We are given the recurrence
\begin{align*}
   g(n) &\leq 9 \cdot g\left(\frac{n}{3}\right) + n^2
\end{align*}

Values of parameters are
\begin{align*}
   a &= 9 \\
   b &= 3 \\
   d &= 2 \\
   s &= 0
\end{align*}

Since \( a = b^d \) that is \( 9 = 3^2 \) the second case applies.
\begin{align*}
   g(n) &\in O\left(n^2 \log n\right)
\end{align*}

\subsection*{(c)}
We are given the recurrence
\begin{align*}
   h(n) &\leq 2 \cdot h\left(\frac{n}{3}\right) + n \log^2 n
\end{align*}

Values of parameters are
\begin{align*}
   a &= 2 \\
   b &= 3 \\
   d &= 1 \\
   s &= 2
\end{align*}

Since \( a < b^d \) that is \( 2 < 3^1 \) the first case applies.
\begin{align*}
   h(n) &\in O\left(n \log^2 n\right)
\end{align*}

\subsection*{(d)}
We are given the recurrence
\begin{align*}
   k(n) &\leq 21 \cdot k\left(\frac{n}{9}\right) + n^{1.5} \sqrt{\log n} \\
   &= 21 \cdot k\left(\frac{n}{9}\right) + n^{1.5} \log^{0.5} n
\end{align*}

Values of parameters are
\begin{align*}
   a &= 21 \\
   b &= 9 \\
   d &= 1.5 \\
   s &= 0.5
\end{align*}

Since \( a < b^d \) that is \( 21 < 9^{1.5} = 27 \) the first case applies.
\begin{align*}
   k(n) &\in O\left(n^{1.5} \log^{0.5} n\right)
\end{align*}

\subsection*{(e)}
We are given the recurrence
\begin{align*}
   \ell(n) &\leq 3 \cdot \ell\left(\frac{n}{9}\right) + n^{0.1} \log n
\end{align*}

Values of parameters are
\begin{align*}
   a &= 3 \\
   b &= 9 \\
   d &= 0.1 \\
   s &= 1
\end{align*}

Since \( a > b^d \) that is \( 3 > 9^{0.1} \) the third case applies.
\begin{align*}
   \ell(n) &\in O\left(n^{\log_9 3}\right) = O\left(n^{0.5}\right) = O\left(\sqrt{n}\right)
\end{align*}

\end{document}