\documentclass{article}
\usepackage{amsmath, amsthm, amssymb}
\usepackage{tikz}
\usepackage{array}
\usepackage{mathtools}
\usepackage{enumitem}
\usepackage{algorithm}
\usepackage{algpseudocode}

\begin{document}
\section*{\huge Homework Sheet 5}
\begin{flushright}
  \begin{tabular}{@{} l l l @{}}
    \textbf{Author} & \textbf{Matriculation Number} & \textbf{Tutor} \\
    Abdullah Oğuz Topçuoğlu         & 7063561 & Maryna Dernovaia \\
    Ahmed Waleed Ahmed Badawy Shora & 7069708 & Jan-Hendrik Gindorf \\
    Yousef Mostafa Farouk Farag     & 7073030 & Thorben Johr \\
  \end{tabular}
\end{flushright}

% Exercise 1 (AVL Trees) 12 points
% Start from an empty AVL tree, and apply the following operations. In your solution, draw the state
% of the AVL tree after each operation:
% ■ Insert(2)
% ■ Insert(3)
% ■ Insert(9)
% ■ Insert(5)
% ■ Insert(8)
% ■ Insert(6)
% ■ Remove(8)
% ■ Remove(5)
% ■ Insert(4)
% ■ Insert(5)
% ■ Remove(6)
% Hint: After Insert(8), the tree looks like the following picture.
% 3
% 2 8
% 5 9
% Exercise 2 (Maximum Height AVL Tree) 4+4 points
% a. Draw an AVL tree of maximum height which consists of 7 nodes and one which consists of 12
% nodes.
% 1
% b. Prove that the AVL tree of height h with the minimum number of nodes has exactly Fh+2 − 1
% nodes.
% Exercise 3 (AVL Trees Extended) 3+4+5 points
% In this exercise, we will augment the AVL tree data structure from the lecture to support more
% operations (in addition to the operations Insert, Remove, Find from the lecture).
% a. Implement the operation Min(h) that given a handle h returns a handle to the node with the
% smallest key currently stored in the sub-tree rooted at h. Your implementation should have
% running time O(log n).
% b. We additionally store in each node of the AVL tree the size of its subtree. Show that we can
% correctly maintain these sizes under the operations Insert and Remove, while maintaining the
% O(log n) running time.
% c. Implement an operation Rank(k) that returns the rank of k, that is, the number of elements x
% in the tree satisfying x.key ≤ k. Your implementation should run in time O(log n).
% Exercise 4 (Next) 4+1+3 points
% For this exercise, we assume that we have augmented the nodes of an AVL tree T such that in each
% node x we maintain a handle x.parent pointing to the parent of this node. The parent handle of the
% root is ⊥. You don’t need to show how to maintain these parent pointers.
% The following procedure computes, given a handle h, the next handle a after h. That is, the
% output a satisfies that akey is minimal among all keys bigger than hkey that are stored in T. If
% no key bigger than hkey is stored in the tree, the output is ⊥.
% 1: procedure Next(Handle h)
% 2: if hright ̸= ⊥ then
% 3: return Min(hright)
% 4: a := h
% 5: while a ̸= ⊥ and akey ≤ hkey do
% 6: a := aparent
% 7: return a
% a. Prove correctness of the procedure, that is that the procedure finds the next handle after h.
% b. Show that the worst case running time is O(log n) in AVL trees.
% c. Show that you can use Next to iterate over the elements stored in a binary search tree in linear
% time. That is, show that the following code has running time O(n).
% 1: h := Min(h)
% 2: while h ̸= ⊥ do
% 3: h := Next(h)

\section*{Exercise 1}
We begin with an empty AVL tree and perform the following operations step by step:
\begin{enumerate}
    \item Insert(2)
    \item Insert(3)
    \item Insert(9)
    \item Insert(5)
    \item Insert(8)
    \item Insert(6)
    \item Remove(8)
    \item Remove(5)
    \item Insert(4)
    \item Insert(5)
    \item Remove(6)
\end{enumerate}

\begin{itemize}
  \item [Insert(2):]
    \begin{center}
      \begin{tikzpicture}[level distance=1.5cm,
        level 1/.style={sibling distance=3cm},
        level 2/.style={sibling distance=1.5cm}]
        \node {2};
      \end{tikzpicture}
    \end{center}
  \item [Insert(3):]
    \begin{center}
      \begin{tikzpicture}[level distance=1.5cm,
        level 1/.style={sibling distance=3cm},
        level 2/.style={sibling distance=1.5cm}]
        \node {2}
          child[missing]
          child {node {3}};
      \end{tikzpicture}
    \end{center}
  \item [Insert(9):]
    \begin{center}
      \begin{tikzpicture}[level distance=1.5cm,
        level 1/.style={sibling distance=3cm},
        level 2/.style={sibling distance=1.5cm}]
        \node {3}
          child {node {2}}
          child {node {9}};
      \end{tikzpicture}
    \end{center}
  \item [Insert(5):]
    \begin{center}
      \begin{tikzpicture}[level distance=1.5cm,
        level 1/.style={sibling distance=3cm},
        level 2/.style={sibling distance=1.5cm}]
        \node {3}
          child {node {2}}
          child {node {9}
            child {node {5}}
            child[missing]};
      \end{tikzpicture}
    \end{center}
  \item [Insert(8):]
    \begin{center}
      \begin{tikzpicture}[level distance=1.5cm,
        level 1/.style={sibling distance=3cm},
        level 2/.style={sibling distance=1.5cm}]
        \node {3}
          child {node {2}}
          child {node {8}
            child {node {5}}
            child {node {9}}};
      \end{tikzpicture}
    \end{center}
  \item [Insert(6):]
    \begin{center}
      \begin{tikzpicture}[level distance=1.5cm,
        level 1/.style={sibling distance=3cm},
        level 2/.style={sibling distance=1.5cm}]
        \node {5}
          child {node {3}
            child {node {2}}
            child[missing]}
          child {node {8}
            child {node {6}}
            child {node {9}}};
      \end{tikzpicture}
    \end{center}
  \item [Remove(8):]
    \begin{center}
      \begin{tikzpicture}[level distance=1.5cm,
        level 1/.style={sibling distance=3cm},
        level 2/.style={sibling distance=1.5cm}]
        \node {5}
          child {node {3}
            child {node {2}}
            child[missing]}
          child {node {9}
            child {node {6}}
            child[missing]};
      \end{tikzpicture}
    \end{center}
  \item [Remove(5):]
    \begin{center}
      \begin{tikzpicture}[level distance=1.5cm,
        level 1/.style={sibling distance=3cm},
        level 2/.style={sibling distance=1.5cm}]
        \node {6}
          child {node {3}
            child {node {2}}
            child[missing]}
          child {node {9}};
      \end{tikzpicture}
    \end{center}
  \item [Insert(4):]
    \begin{center}
      \begin{tikzpicture}[level distance=1.5cm,
        level 1/.style={sibling distance=3cm},
        level 2/.style={sibling distance=1.5cm}]
        \node {6}
          child {node {3}
            child {node {2}}
            child {node {4}}}
          child {node {9}};
      \end{tikzpicture}
    \end{center}
  \item [Insert(5):]
    \begin{center}
      \begin{tikzpicture}[level distance=1.5cm,
        level 1/.style={sibling distance=3cm},
        level 2/.style={sibling distance=1.5cm}]
        \node {4}
          child {node {3}
            child {node {2}}
            child[missing]}
          child {node {6}
            child {node {5}}
            child {node {9}}};
      \end{tikzpicture}
    \end{center}
  \item [Remove(6):]
    \begin{center}
      \begin{tikzpicture}[level distance=1.5cm,
        level 1/.style={sibling distance=3cm},
        level 2/.style={sibling distance=1.5cm}]
        \node {4}
          child {node {3}
            child {node {2}}
            child[missing]}
          child {node {9}
            child {node {5}}
            child[missing]};
      \end{tikzpicture}
    \end{center}
\end{itemize}

\section*{Exercise 2}

\subsection*{(a)}
An AVL tree of maximum height with 7 nodes:
\begin{center}
  \begin{tikzpicture}[level distance=1.5cm,
    level 1/.style={sibling distance=3cm},
    level 2/.style={sibling distance=1.5cm}]
    \node {4}
      child {node {2}
        child {node {1}}
        child[missing]}
      child {node {6}
        child {node {5}}
        child {node {7}
          child[missing]
          child {node {8}}}};
  \end{tikzpicture}
\end{center}
An AVL tree of maximum height with 12 nodes:
\begin{center}
  \begin{tikzpicture}[level distance=1.5cm,
    level 1/.style={sibling distance=4cm},
    level 2/.style={sibling distance=2cm},
    level 3/.style={sibling distance=1.3cm}]
    \node {8}
      child {node {5}
        child {node {3}
          child {node {2}
            child {node {1}}
            child[missing]}
          child {node {4}}}
        child {node {7}
          child {node {6}}
          child[missing]}}
      child {node {11}
        child {node {10}
          child {node {9}}
          child[missing]}
        child {node {12}}};
  \end{tikzpicture}
\end{center}

\end{document}