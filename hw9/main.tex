\documentclass{article}
\usepackage{amsmath, amsthm, amssymb}
\usepackage{tikz}
\usepackage{array}
\usepackage{mathtools}
\usepackage{enumitem}
\usepackage{algorithm}
\usepackage{algpseudocode}

\begin{document}

\section*{\huge Homework Sheet 9}
\begin{flushright}
  \begin{tabular}{@{} l l l @{}}
    \textbf{Author} & \textbf{Matriculation Number} & \textbf{Tutor} \\
    Abdullah Oğuz Topçuoğlu         & 7063561 & Maryna Dernovaia \\
    Ahmed Waleed Ahmed Badawy Shora & 7069708 & Jan-Hendrik Gindorf \\
    Yousef Mostafa Farouk Farag     & 7073030 & Thorben Johr \\
  \end{tabular}
\end{flushright}

% Please note: An exercise that asks you to design or implement an algorithm is to be answered by (1) your
% algorithm in pseudocode, (2) a correctness proof, and (3) a running time analysis.
% Exercise 1 (Chaining and Linear Probing) 3+3 points
% Consider a hash table of size B = 8 and the following sequence of keys k (with hash values h(k)) to
% be inserted.
% k 33 85 11 63 2 94 42
% h(k) 6 2 1 1 7 3 6
% a. Insert the keys into the hash table and resolve collisions by chaining. Draw the state of the hash
% table after every insertion.
% b. Insert the same keys into the hash table and resolve collisions by linear probing. Draw the state of
% the hash table after every insertion.
% Exercise 2 (Random Element) 7+3 points
% Consider a hash table H. In this exercise, we want to design a procedure RandomElement(H) that
% given access to H returns an uniformly random element contained in H in expected time O(1).
% a. Show how to implement RandomElement(H) given that H uses hashing with linear probing
% and has B = Θ(n) buckets.
% b. Why does your approach fail if H uses hashing with chaining?
% Exercise 3 (Fingerprinting) 10 points
% Consider the following problem: Given a matrix M ∈ {0, 1}
% n×n
% represented as a two-dimensional
% array, decide whether M has duplicate rows. Design a randomized algorithm for this problem running
% in expected time O(n
% 2
% ).
% Hint: To use hashing, you have to specify what and how you hash. Beware of collision handling!
% Exercise 4 (Universal Hashing) 4+4+6 points
% Let B be a prime number and U > B. Recall that a hash function h : {1, . . . , U} → {1, . . . , B}
% needs to be initialized before it is used, and the initialization uses randomness. Recall that we call h
% 1
% universal if for any two different keys k, k′ we have P[h(k) = h(k
% ′
% )] ≤ O(1/B) (where the randomness
% comes from the initialization of the hash function). For this exercise, we call h random collision
% avoiding if for two keys k, k′
% chosen uniformly at random, we have P[h(k) = h(k
% ′
% )] ≤ O(1/B) (where
% the randomness comes from the initialization of the hash function as well as the random choice of the
% keys k, k′
% ). Note that universal is a much stronger property than random collision avoiding.
% Also recall that rand(B) generates a random number in {1, . . . , B}. In this exercise, x mod B
% is defined as the unique number y ∈ {1, . . . , B} such that x − y is a multiple of B. Consider the
% following hash functions.
% a. In the initiallization we pick a random number a = rand(B). Define the hash function h by
% h(k) := (ak) mod B.
% Prove that h is random collision avoiding, but not universal.
% b. In the initiallization we pick a random number b = rand(B). Define the hash function h by
% h(k) := (k + b) mod B.
% Prove that h is random collision avoiding, but not universal.
% c. Recall linear hashing from the lecture, and let again wB = ⌊log2 B⌋ as well as d = ⌊log2
% (U)/wB⌋.
% For the initialization, we chose d + 1 numbers a0, . . . , ad ∈ {1, . . . , B} uniformly at random. To
% define the hash function, we split the bits of the key k into parts of width wB, obtaining the
% numbers k0, . . . , kd ∈ {0, . . . , 2
% wB − 1}. We defined
% h(k) :=
% 
% 
% X
% 0≤i≤d
% aiki
% 
%  mod B.
% For this exercise, we now consider the slight variation, where we fix a0 := 1 and only choose the
% numbers a1, . . . , ad uniformly at random. Show that this variation is random collision avoiding as
% well as universal.



\end{document}
